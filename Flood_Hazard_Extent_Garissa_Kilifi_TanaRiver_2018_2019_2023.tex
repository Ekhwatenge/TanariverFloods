% Options for packages loaded elsewhere
\PassOptionsToPackage{unicode}{hyperref}
\PassOptionsToPackage{hyphens}{url}
%
\documentclass[
]{article}
\usepackage{amsmath,amssymb}
\usepackage{iftex}
\ifPDFTeX
  \usepackage[T1]{fontenc}
  \usepackage[utf8]{inputenc}
  \usepackage{textcomp} % provide euro and other symbols
\else % if luatex or xetex
  \usepackage{unicode-math} % this also loads fontspec
  \defaultfontfeatures{Scale=MatchLowercase}
  \defaultfontfeatures[\rmfamily]{Ligatures=TeX,Scale=1}
\fi
\usepackage{lmodern}
\ifPDFTeX\else
  % xetex/luatex font selection
\fi
% Use upquote if available, for straight quotes in verbatim environments
\IfFileExists{upquote.sty}{\usepackage{upquote}}{}
\IfFileExists{microtype.sty}{% use microtype if available
  \usepackage[]{microtype}
  \UseMicrotypeSet[protrusion]{basicmath} % disable protrusion for tt fonts
}{}
\makeatletter
\@ifundefined{KOMAClassName}{% if non-KOMA class
  \IfFileExists{parskip.sty}{%
    \usepackage{parskip}
  }{% else
    \setlength{\parindent}{0pt}
    \setlength{\parskip}{6pt plus 2pt minus 1pt}}
}{% if KOMA class
  \KOMAoptions{parskip=half}}
\makeatother
\usepackage{xcolor}
\usepackage[margin=1in]{geometry}
\usepackage{color}
\usepackage{fancyvrb}
\newcommand{\VerbBar}{|}
\newcommand{\VERB}{\Verb[commandchars=\\\{\}]}
\DefineVerbatimEnvironment{Highlighting}{Verbatim}{commandchars=\\\{\}}
% Add ',fontsize=\small' for more characters per line
\usepackage{framed}
\definecolor{shadecolor}{RGB}{248,248,248}
\newenvironment{Shaded}{\begin{snugshade}}{\end{snugshade}}
\newcommand{\AlertTok}[1]{\textcolor[rgb]{0.94,0.16,0.16}{#1}}
\newcommand{\AnnotationTok}[1]{\textcolor[rgb]{0.56,0.35,0.01}{\textbf{\textit{#1}}}}
\newcommand{\AttributeTok}[1]{\textcolor[rgb]{0.13,0.29,0.53}{#1}}
\newcommand{\BaseNTok}[1]{\textcolor[rgb]{0.00,0.00,0.81}{#1}}
\newcommand{\BuiltInTok}[1]{#1}
\newcommand{\CharTok}[1]{\textcolor[rgb]{0.31,0.60,0.02}{#1}}
\newcommand{\CommentTok}[1]{\textcolor[rgb]{0.56,0.35,0.01}{\textit{#1}}}
\newcommand{\CommentVarTok}[1]{\textcolor[rgb]{0.56,0.35,0.01}{\textbf{\textit{#1}}}}
\newcommand{\ConstantTok}[1]{\textcolor[rgb]{0.56,0.35,0.01}{#1}}
\newcommand{\ControlFlowTok}[1]{\textcolor[rgb]{0.13,0.29,0.53}{\textbf{#1}}}
\newcommand{\DataTypeTok}[1]{\textcolor[rgb]{0.13,0.29,0.53}{#1}}
\newcommand{\DecValTok}[1]{\textcolor[rgb]{0.00,0.00,0.81}{#1}}
\newcommand{\DocumentationTok}[1]{\textcolor[rgb]{0.56,0.35,0.01}{\textbf{\textit{#1}}}}
\newcommand{\ErrorTok}[1]{\textcolor[rgb]{0.64,0.00,0.00}{\textbf{#1}}}
\newcommand{\ExtensionTok}[1]{#1}
\newcommand{\FloatTok}[1]{\textcolor[rgb]{0.00,0.00,0.81}{#1}}
\newcommand{\FunctionTok}[1]{\textcolor[rgb]{0.13,0.29,0.53}{\textbf{#1}}}
\newcommand{\ImportTok}[1]{#1}
\newcommand{\InformationTok}[1]{\textcolor[rgb]{0.56,0.35,0.01}{\textbf{\textit{#1}}}}
\newcommand{\KeywordTok}[1]{\textcolor[rgb]{0.13,0.29,0.53}{\textbf{#1}}}
\newcommand{\NormalTok}[1]{#1}
\newcommand{\OperatorTok}[1]{\textcolor[rgb]{0.81,0.36,0.00}{\textbf{#1}}}
\newcommand{\OtherTok}[1]{\textcolor[rgb]{0.56,0.35,0.01}{#1}}
\newcommand{\PreprocessorTok}[1]{\textcolor[rgb]{0.56,0.35,0.01}{\textit{#1}}}
\newcommand{\RegionMarkerTok}[1]{#1}
\newcommand{\SpecialCharTok}[1]{\textcolor[rgb]{0.81,0.36,0.00}{\textbf{#1}}}
\newcommand{\SpecialStringTok}[1]{\textcolor[rgb]{0.31,0.60,0.02}{#1}}
\newcommand{\StringTok}[1]{\textcolor[rgb]{0.31,0.60,0.02}{#1}}
\newcommand{\VariableTok}[1]{\textcolor[rgb]{0.00,0.00,0.00}{#1}}
\newcommand{\VerbatimStringTok}[1]{\textcolor[rgb]{0.31,0.60,0.02}{#1}}
\newcommand{\WarningTok}[1]{\textcolor[rgb]{0.56,0.35,0.01}{\textbf{\textit{#1}}}}
\usepackage{graphicx}
\makeatletter
\def\maxwidth{\ifdim\Gin@nat@width>\linewidth\linewidth\else\Gin@nat@width\fi}
\def\maxheight{\ifdim\Gin@nat@height>\textheight\textheight\else\Gin@nat@height\fi}
\makeatother
% Scale images if necessary, so that they will not overflow the page
% margins by default, and it is still possible to overwrite the defaults
% using explicit options in \includegraphics[width, height, ...]{}
\setkeys{Gin}{width=\maxwidth,height=\maxheight,keepaspectratio}
% Set default figure placement to htbp
\makeatletter
\def\fps@figure{htbp}
\makeatother
\setlength{\emergencystretch}{3em} % prevent overfull lines
\providecommand{\tightlist}{%
  \setlength{\itemsep}{0pt}\setlength{\parskip}{0pt}}
\setcounter{secnumdepth}{-\maxdimen} % remove section numbering
\ifLuaTeX
  \usepackage{selnolig}  % disable illegal ligatures
\fi
\IfFileExists{bookmark.sty}{\usepackage{bookmark}}{\usepackage{hyperref}}
\IfFileExists{xurl.sty}{\usepackage{xurl}}{} % add URL line breaks if available
\urlstyle{same}
\hypersetup{
  pdftitle={Flood Hazard Extent Analysis for Garissa, Kilifi, and Tana River Counties (2018-2023},
  pdfauthor={Elvira Khwatenge},
  hidelinks,
  pdfcreator={LaTeX via pandoc}}

\title{Flood Hazard Extent Analysis for Garissa, Kilifi, and Tana River
Counties (2018-2023}
\author{Elvira Khwatenge}
\date{2024-09-27}

\begin{document}
\maketitle

This is an R Markdown document. Markdown is a simple formatting syntax
for authoring HTML, PDF, and MS Word documents. For more details on
using R Markdown see \url{http://rmarkdown.rstudio.com}.

When you click the \textbf{Knit} button a document will be generated
that includes both content as well as the output of any embedded R code
chunks within the document. You can embed an R code chunk like this:

Note that the \texttt{echo\ =\ FALSE} parameter was added to the code
chunk to prevent printing of the R code that generated the plot.

Introduction -- Brief description of the dataset and its source. Data
Loading -- Code to load and view the dataset. Data Visualization --
Mapping the flood extent. Exploratory Data Analysis -- Insights,
summaries, and answers to specific questions. Conclusions -- Summary of
findings.

\begin{Shaded}
\begin{Highlighting}[]
\FunctionTok{\#\# Introduction}
\NormalTok{This report analyzes the flood hazard extent data for **Garissa**, **Kilifi**, and **Tana River** counties for the years 2018, 2019, and 2023. The data is in vector format and was merged from various sources, including UNOSAT.}
\end{Highlighting}
\end{Shaded}

\begin{Shaded}
\begin{Highlighting}[]
\CommentTok{\# Load the shapefile}
\NormalTok{flood\_data }\OtherTok{\textless{}{-}} \FunctionTok{st\_read}\NormalTok{(}\StringTok{"/Users/elvir/OneDrive/Desktop/RProjects/TanariverFloods/floodextent01/FloodExtent01.shp"}\NormalTok{)}
\end{Highlighting}
\end{Shaded}

\begin{verbatim}
## Reading layer `FloodExtent01' from data source 
##   `C:\Users\elvir\OneDrive\Desktop\RProjects\TanariverFloods\floodextent01\FloodExtent01.shp' 
##   using driver `ESRI Shapefile'
## Simple feature collection with 23 features and 14 fields
## Geometry type: MULTIPOLYGON
## Dimension:     XY
## Bounding box:  xmin: 38.5286 ymin: -3.872603 xmax: 41.05626 ymax: 0.9863828
## Geodetic CRS:  WGS 84
\end{verbatim}

\begin{Shaded}
\begin{Highlighting}[]
\CommentTok{\# View the first few rows of the data}
\FunctionTok{head}\NormalTok{(flood\_data)}
\end{Highlighting}
\end{Shaded}

\begin{verbatim}
## Simple feature collection with 6 features and 14 fields
## Geometry type: MULTIPOLYGON
## Dimension:     XY
## Bounding box:  xmin: 38.5286 ymin: -3.072294 xmax: 40.72924 ymax: 0.2100281
## Geodetic CRS:  WGS 84
##   Water_Clas Sensor_ID Sensor_Dat Confidence Field_Vali Water_Stat     Notes
## 1          2        44 2023-03-26          2          0          1      <NA>
## 2          1        42 2018-05-04          2          0          1      Bura
## 3          1        42 2018-05-04          2          0          1    Galole
## 4          1        42 2018-05-04          2          0          1    Garsen
## 5          1        42 2018-05-04          2          0          1       Voi
## 6          1        42 2018-05-04          2          0          1 Lamu West
##       Area_m2     Area_ha SenorID_ol StaffID     EventCode  Shape_Leng
## 1   7909970.5   790.99705          0     221 FL20230321KEN  2.05297345
## 2 131527719.9 13152.77199          0     121 FL20180508KEN 31.11923544
## 3  80236864.3  8023.68643          0     121 FL20180508KEN 23.34584839
## 4 189307209.1 18930.72090          0     121 FL20180508KEN 37.46491705
## 5    627544.6    62.75446          0     121 FL20180508KEN  0.03684347
## 6  57172903.8  5717.29038          0     121 FL20180508KEN  0.77677162
##     Shape_Area                       geometry
## 1 6.426164e-04 MULTIPOLYGON (((40.28985 0....
## 2 1.068670e-02 MULTIPOLYGON (((39.93601 -1...
## 3 6.523832e-03 MULTIPOLYGON (((40.16158 -1...
## 4 1.484009e-02 MULTIPOLYGON (((39.22002 -3...
## 5 5.714110e-06 MULTIPOLYGON (((39.21773 -3...
## 6 3.173596e-04 MULTIPOLYGON (((40.71495 -2...
\end{verbatim}

\begin{Shaded}
\begin{Highlighting}[]
\CommentTok{\# Summary of the dataset}
\FunctionTok{summary}\NormalTok{(flood\_data)}
\end{Highlighting}
\end{Shaded}

\begin{verbatim}
##    Water_Clas      Sensor_ID       Sensor_Dat           Confidence   
##  Min.   :1.000   Min.   :42.00   Min.   :2018-05-04   Min.   :0.000  
##  1st Qu.:1.000   1st Qu.:42.00   1st Qu.:2018-05-04   1st Qu.:2.000  
##  Median :1.000   Median :42.00   Median :2018-05-04   Median :2.000  
##  Mean   :1.087   Mean   :42.17   Mean   :2018-08-13   Mean   :1.913  
##  3rd Qu.:1.000   3rd Qu.:42.00   3rd Qu.:2018-05-04   3rd Qu.:2.000  
##  Max.   :2.000   Max.   :44.00   Max.   :2023-03-26   Max.   :2.000  
##    Field_Vali   Water_Stat        Notes              Area_m2         
##  Min.   :0    Min.   :0.0000   Length:23          Min.   :     8541  
##  1st Qu.:0    1st Qu.:1.0000   Class :character   1st Qu.:   691653  
##  Median :0    Median :1.0000   Mode  :character   Median : 14103620  
##  Mean   :0    Mean   :0.9565                      Mean   : 58785913  
##  3rd Qu.:0    3rd Qu.:1.0000                      3rd Qu.: 90822224  
##  Max.   :0    Max.   :1.0000                      Max.   :263335713  
##     Area_ha            SenorID_ol    StaffID       EventCode        
##  Min.   :    0.854   Min.   :0    Min.   :121.0   Length:23         
##  1st Qu.:   69.165   1st Qu.:0    1st Qu.:121.0   Class :character  
##  Median : 1410.362   Median :0    Median :121.0   Mode  :character  
##  Mean   : 5878.591   Mean   :0    Mean   :129.3                     
##  3rd Qu.: 9082.222   3rd Qu.:0    3rd Qu.:121.0                     
##  Max.   :26333.571   Max.   :0    Max.   :221.0                     
##    Shape_Leng         Shape_Area                 geometry 
##  Min.   : 0.00038   Min.   :6.000e-09   MULTIPOLYGON :23  
##  1st Qu.: 0.13742   1st Qu.:2.368e-05   epsg:4326    : 0  
##  Median : 1.25799   Median :3.384e-04   +proj=long...: 0  
##  Mean   : 9.31818   Mean   :3.464e-03                     
##  3rd Qu.:10.54815   3rd Qu.:4.042e-03                     
##  Max.   :46.29957   Max.   :1.889e-02
\end{verbatim}

\begin{Shaded}
\begin{Highlighting}[]
\CommentTok{\# Number of features}
\FunctionTok{nrow}\NormalTok{(flood\_data)}
\end{Highlighting}
\end{Shaded}

\begin{verbatim}
## [1] 23
\end{verbatim}

\begin{Shaded}
\begin{Highlighting}[]
\CommentTok{\# Bounding box of the dataset}
\FunctionTok{st\_bbox}\NormalTok{(flood\_data)}
\end{Highlighting}
\end{Shaded}

\begin{verbatim}
##       xmin       ymin       xmax       ymax 
## 38.5286031 -3.8726029 41.0562588  0.9863828
\end{verbatim}

\begin{Shaded}
\begin{Highlighting}[]
\NormalTok{The map below visualizes the flood extent across Garissa, Kilifi, and Tana River counties over the period 2018, 2019, and 2023.}
\end{Highlighting}
\end{Shaded}

\begin{Shaded}
\begin{Highlighting}[]
\CommentTok{\# Simple plot of the flood extent}
\FunctionTok{plot}\NormalTok{(flood\_data[}\StringTok{"geometry"}\NormalTok{], }\AttributeTok{main =} \StringTok{"Flood Extent in Garissa, Kilifi, and Tana River (2018{-}2023)"}\NormalTok{)}
\end{Highlighting}
\end{Shaded}

\includegraphics{Flood_Hazard_Extent_Garissa_Kilifi_TanaRiver_2018_2019_2023_files/figure-latex/plot-flood-1.pdf}

\begin{Shaded}
\begin{Highlighting}[]
\CommentTok{\# Make geometries valid}
\NormalTok{flood\_data }\OtherTok{\textless{}{-}} \FunctionTok{st\_make\_valid}\NormalTok{(flood\_data)}

\CommentTok{\# Calculate the total flood area (in square meters) for each county}
\NormalTok{flood\_data}\SpecialCharTok{$}\NormalTok{area }\OtherTok{\textless{}{-}} \FunctionTok{st\_area}\NormalTok{(flood\_data)}

\CommentTok{\# Check column names in the dataset}
\FunctionTok{colnames}\NormalTok{(flood\_data)}
\end{Highlighting}
\end{Shaded}

\begin{verbatim}
##  [1] "Water_Clas" "Sensor_ID"  "Sensor_Dat" "Confidence" "Field_Vali"
##  [6] "Water_Stat" "Notes"      "Area_m2"    "Area_ha"    "SenorID_ol"
## [11] "StaffID"    "EventCode"  "Shape_Leng" "Shape_Area" "geometry"  
## [16] "area"
\end{verbatim}

\begin{Shaded}
\begin{Highlighting}[]
\CommentTok{\# Summarize the total flood area by the county{-}equivalent column (e.g., \textquotesingle{}region\textquotesingle{})}
\CommentTok{\# Add a county column manually (this is just an example)}
\CommentTok{\# You would need to assign the correct counties based on the polygons}
\NormalTok{flood\_data}\SpecialCharTok{$}\NormalTok{county }\OtherTok{\textless{}{-}} \FunctionTok{c}\NormalTok{(}\StringTok{"Garissa"}\NormalTok{, }\StringTok{"Kilifi"}\NormalTok{, }\StringTok{"Tana River"}\NormalTok{,}\StringTok{"Garissa"}\NormalTok{, }\StringTok{"Kilifi"}\NormalTok{, }\StringTok{"Tana River"}\NormalTok{,}\StringTok{"Garissa"}\NormalTok{, }\StringTok{"Kilifi"}\NormalTok{, }\StringTok{"Tana River"}\NormalTok{,}\StringTok{"Garissa"}\NormalTok{, }\StringTok{"Kilifi"}\NormalTok{, }\StringTok{"Tana River"}\NormalTok{,}\StringTok{"Garissa"}\NormalTok{, }\StringTok{"Kilifi"}\NormalTok{, }\StringTok{"Tana River"}\NormalTok{,}\StringTok{"Garissa"}\NormalTok{, }\StringTok{"Kilifi"}\NormalTok{, }\StringTok{"Tana River"}\NormalTok{,}\StringTok{"Garissa"}\NormalTok{, }\StringTok{"Kilifi"}\NormalTok{, }\StringTok{"Tana River"}\NormalTok{,}\StringTok{"Garissa"}\NormalTok{, }\StringTok{"Kilifi"}\NormalTok{)  }\CommentTok{\# Example values}

\CommentTok{\# Now summarize the total area by this new \textquotesingle{}county\textquotesingle{} column}
\NormalTok{total\_area\_by\_county }\OtherTok{\textless{}{-}}\NormalTok{ flood\_data }\SpecialCharTok{\%\textgreater{}\%}
  \FunctionTok{group\_by}\NormalTok{(county) }\SpecialCharTok{\%\textgreater{}\%}
  \FunctionTok{summarise}\NormalTok{(}\AttributeTok{total\_area =} \FunctionTok{sum}\NormalTok{(area))}

\CommentTok{\# View the result}
\NormalTok{total\_area\_by\_county}
\end{Highlighting}
\end{Shaded}

\begin{verbatim}
## Simple feature collection with 3 features and 2 fields
## Geometry type: MULTIPOLYGON
## Dimension:     XY
## Bounding box:  xmin: 38.5286 ymin: -3.872603 xmax: 41.05626 ymax: 0.9863828
## Geodetic CRS:  WGS 84
## # A tibble: 3 x 3
##   county     total_area                                                 geometry
##   <chr>           [m^2]                                       <MULTIPOLYGON [°]>
## 1 Garissa    432469901. (((39.47905 -3.872558, 39.47905 -3.872109, 39.47896 -3.~
## 2 Kilifi     240008727. (((39.58443 -3.852975, 39.58443 -3.852795, 39.5842 -3.8~
## 3 Tana River 312174377. (((39.7852 -3.801681, 39.7852 -3.801501, 39.78529 -3.80~
\end{verbatim}

\begin{Shaded}
\begin{Highlighting}[]
\CommentTok{\# Summarize the area by county (assuming there is a county column in the data)}
\NormalTok{total\_area\_by\_county }\OtherTok{\textless{}{-}}\NormalTok{ flood\_data }\SpecialCharTok{\%\textgreater{}\%}
  \FunctionTok{group\_by}\NormalTok{(county) }\SpecialCharTok{\%\textgreater{}\%}
  \FunctionTok{summarise}\NormalTok{(}\AttributeTok{total\_area =} \FunctionTok{sum}\NormalTok{(area))}

\CommentTok{\# View the results}
\NormalTok{total\_area\_by\_county}
\end{Highlighting}
\end{Shaded}

\begin{verbatim}
## Simple feature collection with 3 features and 2 fields
## Geometry type: MULTIPOLYGON
## Dimension:     XY
## Bounding box:  xmin: 38.5286 ymin: -3.872603 xmax: 41.05626 ymax: 0.9863828
## Geodetic CRS:  WGS 84
## # A tibble: 3 x 3
##   county     total_area                                                 geometry
##   <chr>           [m^2]                                       <MULTIPOLYGON [°]>
## 1 Garissa    432469901. (((39.47905 -3.872558, 39.47905 -3.872109, 39.47896 -3.~
## 2 Kilifi     240008727. (((39.58443 -3.852975, 39.58443 -3.852795, 39.5842 -3.8~
## 3 Tana River 312174377. (((39.7852 -3.801681, 39.7852 -3.801501, 39.78529 -3.80~
\end{verbatim}

\begin{Shaded}
\begin{Highlighting}[]
\FunctionTok{colnames}\NormalTok{(flood\_data)}
\end{Highlighting}
\end{Shaded}

\begin{verbatim}
##  [1] "Water_Clas" "Sensor_ID"  "Sensor_Dat" "Confidence" "Field_Vali"
##  [6] "Water_Stat" "Notes"      "Area_m2"    "Area_ha"    "SenorID_ol"
## [11] "StaffID"    "EventCode"  "Shape_Leng" "Shape_Area" "geometry"  
## [16] "area"       "county"
\end{verbatim}

\begin{Shaded}
\begin{Highlighting}[]
    \CommentTok{\# Summarize the total flood area (in hectares) by county}
\NormalTok{total\_flood\_by\_county }\OtherTok{\textless{}{-}}\NormalTok{ flood\_data }\SpecialCharTok{\%\textgreater{}\%}
  \FunctionTok{group\_by}\NormalTok{(county) }\SpecialCharTok{\%\textgreater{}\%}
  \FunctionTok{summarise}\NormalTok{(}\AttributeTok{total\_area\_ha =} \FunctionTok{sum}\NormalTok{(Area\_ha, }\AttributeTok{na.rm =} \ConstantTok{TRUE}\NormalTok{))}

\CommentTok{\# View the results}
\NormalTok{total\_flood\_by\_county}
\end{Highlighting}
\end{Shaded}

\begin{verbatim}
## Simple feature collection with 3 features and 2 fields
## Geometry type: MULTIPOLYGON
## Dimension:     XY
## Bounding box:  xmin: 38.5286 ymin: -3.872603 xmax: 41.05626 ymax: 0.9863828
## Geodetic CRS:  WGS 84
## # A tibble: 3 x 3
##   county     total_area_ha                                              geometry
##   <chr>              <dbl>                                    <MULTIPOLYGON [°]>
## 1 Garissa           70038. (((39.47905 -3.872558, 39.47905 -3.872109, 39.47896 ~
## 2 Kilifi            26689. (((39.58443 -3.852975, 39.58443 -3.852795, 39.5842 -~
## 3 Tana River        38481. (((39.7852 -3.801681, 39.7852 -3.801501, 39.78529 -3~
\end{verbatim}

\hypertarget{conclusion}{%
\subsection{Conclusion}\label{conclusion}}

This report analyzed flood hazard extent data for \textbf{Garissa},
\textbf{Kilifi}, and \textbf{Tana River} counties over the years 2018,
2019, and 2023. The analysis provided insights into the total flood
areas for each county, the changes in flood extent over time, and
highlighted which county was most affected by flooding in 2023.

Further analysis could focus on the specific causes of the flooding, its
impact on local communities, and possible mitigation strategies.

\end{document}
